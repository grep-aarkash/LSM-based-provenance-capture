\section{Related Work}
There are many types of tools dedicated to the analysis of the kernel. The most basic method in order to find errors faster, is to look for reasons in the code known to be symptomatic of certain classes of problems in the kernel.
\subsection{Information flow control systems}
Previous work on information flow control enforcement at the OS level, such as HiStar [41],
Flume [42], and Weir [43], uses labels to define security and integrity contexts that constrain information flows between kernel
objects. Labels map to kernel objects, and a process requires decentralised management capabilities to modify its labels. Point-topoint access control decisions are made to evaluate the validity
of an information flow. Through transitivity, it is possible to express constraints on a workflow (e.g., collected user information
can only be shared with third parties as an aggregate). SELinux [44]
provides a similar information flow control mechanism but without decentralised management. A typical way of representing and
thinking about information flow in a system is through a directed
graph. However, current object labelling abstractions do not take
advantage of this representation, and it is difficult to reason about
when defining policies. CamQuery differs from these systems in that it allows the implementation of such mechanisms directly on the
graph abstraction.