
\documentclass{IEEEtran}
\IEEEoverridecommandlockouts
% The preceding line is only needed to identify funding in the first footnote. If that is unneeded, please comment it out.
\usepackage{cite}
\usepackage{amsmath,amssymb,amsfonts}
\usepackage{algorithmic}
\usepackage{graphicx}
\usepackage{textcomp}
\usepackage{xcolor}
\def\BibTeX{{\rm B\kern-.05em{\sc i\kern-.025em b}\kern-.08em
    T\kern-.1667em\lower.7ex\hbox{E}\kern-.125emX}}


\begin{document}

\title{Linux security modules and whole-system provenance capture\\
{\footnotesize}
\thanks{}
}

\author{\IEEEauthorblockN{ Aarti Kashyap} \\
\IEEEauthorblockA{\textit{Electrical and Computer Engineering} \\
\textit{ University of British Columbia}\\
Vancouver, Canada \\
kaarti.sr@gmail.com}


}

\maketitle
 \textit{“A process cannot be understood by stopping it. Understanding must move with the flow of the process, must join it and flow with it.” } - Frank Herbert
\section*{Abstract}
    
     \vskip 0.2in
	 \textbf{Data provenance describes how data came to be in its present form. It includes data sources and the transformations that have been applied to them. There have been several different OS provenance capture tools in the past. However, only CamFlow and its predecessor Linux Provenance Modules use Linux Security Modules that is a  framework that allows the Linux kernel to support a variety of computer security models while avoiding favouritism toward any single security implementation. This paper examines the relationship between LSM and Camflow. The two key research questions that are addressed: 1) Does
	 the last stable version of Linux kernel(v4.20) capture all information flows? 2) Given the results of (1) and the data captured by a whole-system provenance capture mechanism utilizing LSM hooks, can we prove that an intrusion will be reflected as an anomaly in the provenance graphs. The part 1 of our work focuses on ensuring that every information flow  in v4.2 goes through a LSM hook. This is an important question that helps in validating the use of kernel provenance for intrusion detection. We found two system calls which have flows that don't pass through the LSM hooks. }

\section{Introduction}



\label{What is provenance im general and with respect to computing?}

A provenance-aware system automatically gathers and
reports metadata that describes the history of each object being processed on the system. This allows users to
track, and understand, how a piece of data came to exist in its current state. The application of provenance
is presently of enormous interest in a variety of disparate communities including scientific data processing,
databases, software development, and storage [33, 34].
Provenance has also been demonstrated to be of great
value to security by identifying malicious activity in data
centers [35, 36, 37], improving Mandatory Access
Control (MAC) labels [38,39,40], and assuring regulatory compliance [3].
\vskip 0.1in



Data provenance can help detect such intrusions in the kernel. It only provides with the capability to detect intrusions not prevent them. However, in order to make sure that all the intrusions are getting detected, we need a way to capture the complete data flow in the system which is why we chose to work with Camflow. Camflow is a practical implementation of whole-system provenance capture that can be easily maintained and deployed. They use Linux security modules and Netfilter hooks as the underlying framework to capture the data flows which provides us the ability to capture all the information flows. This ability to capture all information flows in a system is defined as whole-system provenance capture. 
\vskip 0.1in

\label{What are we doing?}
In this paper, we examine if the whole system provenance capture mechanism developed by Camflow can be utilized for intrusion detection. Camflow uses exisiting capture techniques provided by the functionalities of Linux operating systems. The work we have undertaken focuses on information security, and more specifically on the two security properties which should hold throughout the system. 
\vskip 0.1in
\begin{enumerate}
	\item  Confidentiality: Only authorized users should have access to the information. In order to make sure that only authorized users have access to certain files in the Linux operating system, a general lightweight access control framework was developed. This was the Linux security modules(LSM) project(25,26). A number of existing acess control implementations including SELinux(27) were adapted to use LSMs.
	\vskip 0.1in
	\item Integrity The information stored in the system should not undergo any change. The crash of a PhD thesis in progress in word without any data recovery or the unauthorized creation of a user account in service by a hacker are some examples of intolerable corruptions. 
\end{enumerate}
\vskip 0.1in
In order to meet the needs of confidentiality and integrity, system administrators assign number of security policies. Hence, each user permissions and object of the system contains an information security level which limits the access. Object in this case is defined as the abstraction of the system which may contain information. 

%\vskip 0.1in
%\label{Applications of data provenance}
%
%Data provenance has wide range of applications ranging from dependability ( reliability and security) of the system to reproducibility of computational experiments.
%\vskip 0.1in
\label{Role of data povenance in security}

Security is a major concern since there is no permanent fix to detect intrusions. Its a race between attackers and defenders.
An example to show that the kernel is still under threat Xioo et al.[22] showed that attackers are able to manipulate the running behaviours of operating systems without injecting any malicious code. This type of an attack is called as kernel data attack. With the power of tampering data, the attackers can stealthily subvert various kernel security mechanisms. These policies express security needs. Hence the first part of our work focuses on one of these mechanisms: monitoring information flow to ensure that the integrity property holds. 
\vskip 0.1in
\label{Challenge 1}

We first focus on a methodology proposed by Georget et al.[3] to verify if indeed every security related flow goes through an LSM hook. The methodology proposed by Georget et al. had been designed for linux kernel v4.3 to ensure that all security flows are passing through the hooks. We guarantee the same for v4.20 which is the last formal release. The tools developed by Georget et al.[3], however, can be utilized for performing static analysis with any kernel version. A second challenge in determining if the use of kernel provenance can help in intrusion detection, we prove that a security breach is reflected in the provenance graph it produces. 
\vskip 0.1in

The key contributions of our work include: 1) ensuring if all information flows pass through the Linux security modules hooks. This work focuses on understanding if the placement of hooks is correct. 2) if Camflow uses the LSM and Netfilter hooks to capture the data, in case of intrusions the provenance graph is able to capture it. 

\section{Background and Objectives}
\begin{figure}
	\centering
	\includegraphics[width=0.7\linewidth]{graph}
	\caption[Provenance graph]{Process 1 clones process 2. Process 2 writes to a pipe. Process 1 read from the same pipe}
	\label{fig:graph}
\end{figure}
\begin{figure}
	\centering
	\includegraphics[width=0.7\linewidth]{Annotated-provenance-graph}
	\caption[Provenance graph]{Annotated provenance graph}
	\label{fig:annotated-provenance-graph}
\end{figure}

To provide context for the rest of the paper, we first introduce a few concepts. We introduce the notion of whole-system provenance, provenance graphs, Linux security modules and complete mediation property.

\subsection{Whole-system provenance capture}
\label{A little about the graphs}
Data provenance was originally introduced to understand the origin of data in a database. [17, 18] According to the W3C [19] standards provenance is defined as a directed acyclic graph (DAG). The vertices in the DAG represent entities(data), activities(transformations of data) and agents(persons or organizations). The edges represent the relationships between these elements. Mapping the graph definition to our system, which is the OS level entities are kernel objects, such as messages, network packets, files etc., but also xattributes, inode attributes, exec call parameters, network addresses, etc.
Activities are the tasks or processes carrying out manipulations on entities causing data flows. The agents are the persons or the organizations, who control the activities on different entities. These are the users and the groups at the OS level. 
 the agents are users and groups. Fig 1 illustrates these concepts. In the example in Fig 1, process 1 clones process 2. Process 2 writes to a pipe and finally process 1 read from the same pipe.
\vskip 0.1in 
Processes exchange information via system calls. Some system calls represent information exchange at some discrete point in time e.g, read, write; others may create shared states, e.g, nmap.


\label{A small recap into data provenance and whole system provenance capture}
\vskip 0.1in
There have been multiple whole system provenance capture systems proposed in the past such as HiFi[11], PASS[10]. However, they had a couple of problems: 1) struggled to keep abreast with current OS releases 2) Did not have whole system provenance capture guarantees 3) generated too much data 4)imposed too much overhead. Learning from the lessons from the past the whole system provenance capture systems, Camflow was introduced which promised to resolve all the above mentioned issues. Camflow addressed the above mentioned shortcomings 1) by leveraging the latest kernel defining to achieve efficiency 2) using a self-contained, easily maintainable implementation relying on Linux Security Modules, Net-filter, and other existing kernel facilities.


\subsection{Linux Security Modules}
\label{LSMs}
Since the kernel version 2.6, Linux added support for a framework to implement security extensions for the kernel called Linux Kernel Modules(LSM)(14). This framework provides a set of hooks strategically
placed in the kernel code associated with felds in internal data
structures for exclusive use by security extensions. The hooks are
functions which can be used by security extensions to (1), allocate,
free, and maintain the security state of various internal data structures having a dedicated security field, and (2), implement security
checks at specific points of execution, based on the security state
and a policy.  Security modules have a chance to apply security
restrictions anywhere a hook is present, but only at these places.
LSM’s original design is the access control and this has dictated the
placement of hooks in the code. . It is thus necessary to verify the
correctness of this placement for the purpose of information flow
tracking to ensure that information flow trackers. 
\vskip 0.2in
Since, Camflow uses LSMs and Net-Filters to capture the system provenance, the need to verify the correct placement of hooks for Camflow becomes necessary. 



\subsection{Complete Mediation Property}
The first goal of our contribution is to verify the property called "Complete Mediation property". According to this property for any execution path in the kernel starting with a system call and leading to an information flow, there is at least one LSM hook in the path which is reached before the flow is performed. It's important to verify this property. The reason is because if there exists a path generating a flow but not going through any LSM hooks, then there exists an opening for a malicious program to perform illegal actions without triggering any alarms. This is because information flow monitor can only react when one of the hooks is reached. 
\vskip 0.1in
In order to identify all the paths which lead to information flows requires solving two common problems in static analysis. The first problem is that the number of execution paths is infinite because of loops and recursions in the code. In order to finitize the code a subset is selected which should be sufficient to draw a conclusion for all the paths. In other words constructing an abstraction of the code which can be mapped to the concrete code after the analysis is required. The second problem  is that many execution paths that appear in the control flow graph (CFG) cannot actually be taken. These are called as the infeasible paths[20].  
\vskip 0.1in
For our analysis we consider all the system calls in the kernel version 4.20. Flow control requires knowing precisely when an information flow starts and when it stops. If this information is not available, it is not possible to maintain a correct representation of all flows currently taking place in the system at any given time. Since LSM was designed with access control in mind, some hooks might be missing to perform information flow tracking in every kernel update that comes. However, using static analysis we can find if some hooks are missing to perform information flow tracking. 
%\textbf{Georget et al methodology}
\vskip 0.1in
\label{Information flow}
The purpose of information flow control is to monitor the way in which information is disseminated in the system once it is out of its original container. This is unlike access control which can only enforce rules on how whose containers are accessed. Several scientific and technical challenges exist in ensuring complete information flow. One of them being the large Linux kernel code base. Georget et al.[3] tackles this issue in his work.

\vskip 0.1in
The most common way to meet the security objectives is access control. In order to ensure this, each level of security permissions is associated with the read or modifications of objects assigned to this level. This makes sure that only authorized users can read or alter information when stored in the object marked at the correct level of security. However, the fragility of this approach lies in the fact that once the information is out of their original container, the policy can no longer protect the information. To fully protect the information, the policy must be transitive. If Alice has access to a file but Bob does not, then we should make sure that Alice does not have some means to communicate with Bob. If Alice passes the information, even accidentally, then that is the violation of the confidentiality system.

\vskip 0.1in
The information flow control responds to the above problem. By keeping a track of information movement that took place between objects in the history of the system, it is able to protect information even when outside of its original container. This is not limited as the communications in the case of access control. As per the above Alice and Bob example, Alice has the right to contact Bob, until the knowledge to which Bob does not have access to is communicated. If that is indeed the case, then the communication has to stop. 

Hence, the initial Linux security module framework was built to support the access control mechanisms. We want to ensure that the framework is suitable enough to implement information flow mechanisms. 

%\label{Georget methodology}
%In order to improve the state of art of the information flow systems, Georget et. al developed a  plugin for the GCC compiler to easily extract and visualize control flow graphs of kernel functions...


\label{Camflow}
Camflow which utilizes LSM for the whole-system provenance capture. It collects the provenance data and constructs provenance graphs from the collected data. Now that we are aware that Georget's methodology ensures the placement of hooks such that complete information flow is possible, we show that the violations are reflected in the provenance graphs.


\begin{figure*}
	\centering
	\includegraphics[width=0.7\linewidth]{camflow}
	\caption[Architecture overview]{Camflow architure}
	\label{fig:camflow}
\end{figure*}
\begin{figure}
	\centering
	\includegraphics[width=0.7\linewidth]{LSM-hook}
	\caption[LSM hook]{LSM hook working for system call "open"}
	\label{fig:lsm-hook}
\end{figure}


\label{Introduction}
\section{LSMs and LSM hooks}
Though we have introduced what a Linux Security Module(LSM) means, we will discuss it in a little more detail. The Linux Security Module (LSM) framework provides a mechanism for various security checks to be hooked by new kernel extensions. The name “module” is a bit of a misnomer since these extensions are not actually loadable kernel modules. Instead, they are selectable at build-time via CONFIG\_DEFAULT\_SECURITY and can be overridden at boot-time via the "security=..." kernel command line argument, in the case where multiple LSMs were built into a given kernel.
\subsection{LSMs and policies}
The primary users of the LSM interface are Mandatory Access Control (MAC) extensions which provide a comprehensive security policy. Examples include SELinux, Smack, Tomoyo, and AppArmor. In addition to the larger MAC extensions, other extensions can be built using the LSM to provide specific changes to system operation when these tweaks are not available in the core functionality of Linux itself.
\subsection{Types of LSMs}
Without a specific LSM built into the kernel, the default LSM will be the Linux capabilities system. Most LSMs choose to extend the capabilities system, building their checks on top of the defined capability hooks.
\vskip 0.1in
The different types of Linux Security Modules are listed in Table 1. The entire list of the LSMs can be found by reading \textit{/sys/kernel/security/lsm}. The Table 1 reflects the order in which checks are made. The capability module is always first, followed by "minor" e.g. Yama) and then the one “major” module (e.g. SELinux) if there is one configured. This information will help us in understanding the next part of our research. 
\subsection{LSM Hooks}
In order to explain how a LSM hook works, we show an example in Fig 4. We take system call open as an example. We can see in Fig 4 that just before the kernel addresses an internal object, a check function provided by LSM is called. Hence, LSM allows the modules to understand wheather subject S is allowed to perform an action OP over kernel's internal object OBJ.


\begin{table}[ht]
	\caption{Linux Security Modules}
	\centering
	\begin{tabular}{c}
		\hline\hline 
		Different LSMs \\
		\hline
		AppArmor \\
		LoadPin  \\
		SELinux  \\
		Smack   \\
		TOMOYO \\
		Yama \\
	\end{tabular}	
\end{table}


\section{Modelling the system calls causing flows}
Analysis of programs aims verification of certain properties expected of them. Conventionally, there are two categories of anlysis, dynamic analysis and static analysis. Static analysis checks the properties on the souce code of the program, according to the declared semantics of the programming language, without the need to compile the program into machine language. 
The analysis proposed relies on the C compiler from the
Gnu Compilers Collection [23], used to compile the Linux
kernel.

\subsection{Control flow graphs}
The analysis technique we use if has been proposed by Georget et al. () for a subset of the system calls. It's a four step methodology which relies on the C compiler from the Gnu Compilers Collection(21). 

\begin{enumerate}
	\item The model designed by Georget to represent system calls and their execution paths does not describe the C source code. They instead use an internal representation called GIMPLE[]. 
	\item Each system call is represented by a control flow graph (CFG).
	\item The paths in these graphs model the execution paths in the program as defined by the classical graph theory[].
	\item The system calls are analysed one at a time. 
	\item Each system call contains multiple functions. These functions are inlined into the system calls to reduce the analysis to intra-procedural case. 
	\item Finally, in the CFGs, two kinds of nodes are marked: the nodes which correspond to the LSM hooks and the nodes which correspond to operation which generate the flows. 
	\end{enumerate}

\subsection{Constraints in modelling}
In the CFGs which we construct , a node is not a basic block but a simple GIMPLE instruction. The analysis methodology does not deal with all expressions and variables of the language. Another reason for the same is that usage of floating-point values is explicitly prohibited in the Linux code. Variables representing structures or unions are also not handled when they involve pointer arithmetic. Global or volatile variables are also not handled, since they can have an arbitrary value at any point in the execution. Ignoring some variables does not hinder the soundness of
our approach: less impossible paths might be detected as such but
we never declare as impossible a possible path. A path in the CFG is said to be impossible when any execution
that would follow it would enter in an impossible state. For example,
a path including two conditional branching with incompatible
conditions would require a Boolean expression to be both true and
false at the same time. 
\vskip 0.1in
There are powerful static analysers available such as Blast(24), available. However, for our need to ensure the complete mediation property, we don't need a framework which deals with complex types and data structures. Torvalds, creator and maintainer-in-chief of Linux has also developed a semantic analyzer for C called Sparse(27).
\vskip 0.1in
The mediation property which we have described in the previous section despite introducing the above mentioned constraints provides us with precise modelling.

\section{Static analysis on paths}
The goal of static analysis is to verify that information flow goes through a LSM hook or that it is impossible. In order to do so, we need to find the information flows for any CFG representing a system call.

\subsection{Verifying complete mediation}
We consider the set \textit{Paths} of all paths in a CFG.We introduce two particular subsets in this:(1) the set
Paths$_{flows}$  of paths starting at the initial node of the CFG and
ending at one node generating an information flow; and (2) the
set Paths$_{LSM}$  of paths having a node corresponding to a LSM
hook. The placement of the hooks would be obviously correct
if we could prove that Paths$_{flows}$ $\subseteq$ Paths$_{LSM}$ . However, as
we will see, this is not the case. Pf =  Paths$_{flows}$ \ Paths$_{LSM}$. The sets involved in the analysis are shown diagrammatically in Fig. 5.
represents the set of paths that may be problematic.
\vskip 0.1in
As explained earlier, some paths in Paths$_{flows}$  are actually
impossible, and therefore even if there are no LSM hooks
in them, they are not actually problematic. Recall that an
impossible path is a path that does not correspond to a possible
execution. The objective of our analysis is thus to verify that
P$_f$ $\subseteq$ I where I $\subseteq$ Paths is the set of impossible paths in
the CFG.
\vskip 0.1in
The property which we are looking to verify is the complete mediation property. Explaining the entire proof is out of scope of this paper. Hence, we state the entire property here.
\vskip 0.1in
\subsubsection[Property 1 (Complete Mediation)]{Property 1}\textit{(Complete Mediation)}
\vskip 0.2in
\textit{Complete mediation holds iff: P$_f$ $\subseteq$ I, i.e. all the execution
	paths that perform an information flow and are not controlled
	by the information flow monitor since they do not contain a
	LSM hook are impossible according to the static analysis.}
\begin{figure}
	\centering
	\includegraphics[width=0.7\linewidth]{Sets-involved}
	\caption[Sets involved in analysis]{Sets involved in analysis}
	\label{fig:sets-involved}
\end{figure}

\subsection{Soundness of proofs}
The soundness of our analysis is stated as follows.
\vskip 0.1in
\subsubsection{Theorem 1 }\textit{(Soundness)}
\vskip 0.1in

\textit{For each path p in a CFG, for all concrete configurations $\theta_1$
	and $\theta_2$, and all abstract configurations k$_1$ and k$_2$ such that
	$\theta_1$ $\rightarrow*_p$ $\theta_2$ and 
	k$_1$ $\rightarrow*p$ k$_2$, we have $\theta_1$ $\vDash$ k$_1$ $\Longrightarrow$ $\theta_2$ $\vDash$ k$_2$}
\vskip 0.2in
In simple words, according to the soundness property, executable paths are defined as the set of all paths p such that there exist two concrete configurations $\theta_1$ and $\theta_2$ such that $\theta_1$ $\rightarrow_p$ $\theta_2$. A transition may not exist from on configuration to another if, along the path p, there is an edge bearing a condition which is not verified by the memory state. e. The set of impossible paths is defined as the set of paths p for which
given any two abstract configurations k$_1$ and k$_2$, if k$_1$ $\rightarrow_p*$ k$_2$, then
$\nvDash$ k$_2$. In other words, there is no satisfiable configuration that can
result from the analysis of path p. The following proposition is an
immediate consequence of the previous one.
\vskip 0.2in

\subsubsection{Theorem 2 }\textit{(Executable vs Impossible paths)}
\vskip 0.1in
\textit{The sets of
	executable paths and impossible paths are disjoint.}

\subsection{Implementation and results}
The static analysis is run during the compilation itself. To this effect, we use the plug-ins developed by Georget et al. We also followed the same methodology as proposed by Georget et al. in the previous kernel version 4.3. We performed the experimentation for kernel version 4.20 which is the last stable release as of now. 
\vskip 0.2in
We use a plugin developed by Geroget () for GCC version 4.8.  This plugin is
inserted into the compilation process, when the code is in GIMPLE
form. We chose to place the plugin as late as possible in the compilation process, just before GCC abandons the graph intermediate
representation, in order to benefit the most from optimizations
and to work on a code as simple as possible. This allows us to do
the assumptions we presented in the previous sections, such as
the particular properties of loops. The representation we dump is
broken down to very elementary pieces. The code we handle is in
three-addresses mode, which has the effect that all complex boolean
conditions that could exist in the original code base are already
broken down in the appropriate number of binary decision nodes
and branching. We also leverage as much as possible the compiler.
For example, we rely on it to identify the loops in the CFG, to know
the precise typing information of variables, and to run the points-to
analysis on pointers. The latter is already implemented because it
is used for several optimizations passes by GCC. Finally, the careful
use of inlining and the fact that GCC applies some optimizations
on the code actually limits the path explosion. We make use of the tools available on the kayrebt website (31).

In order to do so, we first look at the system calls difference between kernel version 4.3 and 4.20. In total we identify 460 system calls in the stable release 4.20. But from the observation, we can also notice that not all system calls are capable of generating flows. Understanding the difference between the system calls between different kernel versions, itself is a separate study. The previous study was done by Hauser (32) for the kernel version 4.7 for his doctoral thesis. New system calls are added and used as per the applications requirements. 
\vskip 0.2in
Prior to running analysis,we first need to place a special annotations on tplaces in the code of system calls where information flows are performed. When the analysis is run on a system call, for each
annotated places, a subgraph of the CFG built by GCC is considered:
the subgraph of paths starting from the node representing the entry
of the system call and going to the information flow node that
do not pass through any LSM hook. These paths are the set P$_f$
.
For each path, we start with an empty abstract configuration and
we update the configuration as we go along the path as per the
transition rules of the abstract semantics. The satisfiability of the
abstract configuration is tested by Yices [5], a SAT-solver equipped
with a decidable subset of the classic theory of integers. If the
constraint solver declares the set of constraints unsatisfiable then
we declare the path impossible.
\vskip 0.2in
In order to understand and visualize the information flows, we use the tools available on http://kayrebt.gforge.inria.fr/. We categorize the flows into two types discrete flow and continuous flow as shown in the Table 2, 3 and 4. Table 2 shows a summary of the discrete flows for the different types of system calls. We have similar flows recorded for the write, send, receive system calls. Table 3 shows the system calls which were updates in the kernel version 4.20. However, Georget did a similar analysis and found the missing LSM hooks for 29 system calls which were updated from kernel version 3.2 to 4.3. Table 4 lists the system calls from version 4.3 to 4.20 whose information flows which are discrete do not trigger all LSM hooks. Table 5 shows system calls added from version 4.3 to 4.20 which do not trigger all LSM hooks when called. These system calls have continuous flows. This means that the flow is from both sides, the file to the memory and vice versa. However, we notice that the LSM hook is triggered only once during the exchange.

\begin{table}[ht]
	\caption{Read system calls}
	\centering
	\begin{tabular}{c c}
		\hline\hline 
	System calls & Discrete flow  \\
		\hline
		read & File $\rightarrow$ memory of the calling process \\
			readv & File $\rightarrow$ memory of the calling process \\
				preadv & File $\rightarrow$ memory of the calling process \\
					pread64 & File $\rightarrow$ memory of the calling process \\

	\end{tabular}	
\end{table}


\begin{table}[ht]
	\caption{Updated system calls in version 4.20}
	\centering
	\begin{tabular}{c c}
		\hline\hline 
		System calls & Discrete flow  \\
		\hline
		vmsplice. & Memory of the calling process $\rightarrow$ tube \\
		 & tube $\rightarrow$ memory of the calling process \\
		process\_vm\_readv & Memory of another process  \\
	 & $\rightarrow$ memory of the calling process \\
	 	process\_vm\_writev & Memory of another process  \\
	 & $\rightarrow$ memory of the calling process \\
		
	\end{tabular}	
\end{table}

\begin{table}[ht]
	\caption{Updated system calls in version 4.20 which don't trigger LSM hooks but have discrete flows.}
	\centering
	\begin{tabular}{c c}
		\hline\hline 
		System calls & Discrete flow  \\
		\hline
		migrate\_pages & Memory of another process  \\
		& $\rightarrow$ memory of the calling process \\
		move\_pages & Memory of another process  \\
		& $\rightarrow$ memory of the calling process \\
		
		
	\end{tabular}	
\end{table}

\begin{table}[ht]
	\caption{Updated system calls in version 4.20 which don't trigger LSM hooks but have discrete flows.}
	\centering
	\begin{tabular}{c c}
		\hline\hline 
		System calls & continuous flow  \\
		\hline
	
		mmap\_pgoff. & Regular file or device  \\
		& $\leftrightarrow$ memory of the current process \\
		& regular file or device  \\
		& $\rightarrow$ memory of the current process \\
		
	\end{tabular}	
\end{table}






\section{Camflow}
We have given a little introduction to Camflow in the previous sections. The Camflow[6] builds upon and learns from previous
OS provenance capture mechanisms, namely PASS, Hi-Fi,
and LPM. The provenance data is captured through Linux
Security Module hooks and NetFilter hooks. The provenance data is transferred to the user space through relayfs,
where it can be stored or analysed. Applications can enrich system level provenance with application-specific details
through a pseudo-file interface. The provenance capture can
be tailored to suit the needs of the application. This is
done through pseudofiles and restricted to the owners of the
capability CAP\_AUDIT\_CONTROL. Camflow provides a library that,
through an API, abstracts interactions with the pseudo-files
and relayfs. This is shown in Fig. 3. 
\vskip 0.1in
CamFlow records how information is exchanged within a system
through system calls. Some calls represent an exchange of information at a point in time e.g., read, write; others may create shared
state, e.g., mmap. The former can easily be expressed within the
PROV-DM model, the latter is more complex to model.
\begin{figure}
	\centering
	\includegraphics[width=0.7\linewidth]{provenance-g}
	\caption{An example of a provenance graph. CamFlow-Provenance partial graph example. A process P reads information from a file F, sends the information
		over a socket S, and updates F based on information received
		through S}
	\label{fig:provenance-g}
\end{figure}

\begin{figure}
	\centering
	\includegraphics[width=0.7\linewidth]{provenance-capture}
	\caption[open system call]{Executing an open system call. In green is the capture mechanism. The pink is the provenance tailoring mechanism.}
	\label{fig:provenance-capture}
\end{figure}
\subsection{Provenance Data Model}
As explained in the previous sections according to W3C standards the provenance is defined as a directed acyclic graph(DAG). Fig 6 represents an example of a provenance graph. A process P reads information from a file F, sends the information
over a socket S, and updates F based on information received
through S.
\section{Limitations}
The limitations of our approaches are that we ignore the side channel attacks. We do not consider timing attacks. We restrict ourselves
to explicit information flows such as information storing or interprocess communications through means designed for this purpose.
We thus exclude covert channels and information leakage due to
some event or operation not occurring in the system.
\section{Related Work}
There are many types of tools dedicated to the analysis of the kernel. The most basic method in order to find errors faster, is to look for reasons in the code known to be symptomatic of certain classes of problems. For example, to detect the omission of a deallocating memory in a function or a direct comparison between a
pointer and an integer. In response to this problem has been
developed Ladybug [ 28]. MECA [ 29 ] and CQUAL [ 30 ] are other tools which were developed for performing the static analysis of the kernel.
\section{Conclusion}
Hence, finally we run the analysis which was introduced by Georget et al[3] for the version 4.3 and we find three system calls which call LSM hooks but not all paths are mediated. These are listed in table 4 and 5.
\vskip 0.2in
We further provide a formalism to show that if all the flows pass through LSM hooks,then the flows gets captured by CamFlow. If they get captured by CamFlow, and assuming that no data is lost during the construction of the graphs, all the anomalies show up in the graphs.
\section{Discussion}
The first interesting discussion point is under-
standing the completeness-security gap if any. Cur-
rently, the DAGs are generated with respect to the
provenance data. However, this is not taking the
timing aspect into consideration. Provenance cap-
ture mechanisms currently are data related. The way
we can capture anomalies is by learning the data
pattern from previous immutable data. However, if
the attacker brings in the timing aspect, keeping the
data intact can that cause damage? It depends on
the application if data delays can cause harm.
Another interesting point to discuss would be the
completeness of information flows means that it
guarantees that all flows are being captured. So
you need to place hooks at the relevant places
which ensures that all information flow paths are
being recorded. However, again this does not cover
the event aspect of the system. What if instead of
modifying the data with alone or modifying the data
with respect to time, it modifies the event in a smart
way such that the correlations created by the DAG
cannot spot it. Instead of sending the data from
process P to process Q, it sends it to process R.

%
%
%
%
%\section{Introduction}
%System security is a race between the attackers and the defenders. The attackers adapt their attack model based on the defense mechanisms deployed on systems. The designers of the systems can build a completely correct system using formal methods such as theorem proving and model checking \cite{b4}. However, this only proves the correctness properties of the systems such as " Making sure that the system is following the correct protocol " or  " the shared memory allocation is done efficiently". The attacker can make sure that there is no violation in these properties and still manage to attack the system. In order to beat the attacker in this game, security-based mitigation techniques are proposed, which lack complete security coverage. \cite{b5}. Hence, in order to obtain a full or wider coverage of the system, using provenance-based techniques is an ideal way to proceed. \cite{b6}
%
%Provenance has many different definitions when used in different contexts. The simplest way to define provenance is a formal set of documents, to understand the beginning of something's existence and origin. These documents can be used to guide the authenticity or quality of the item. In a computing context, data provenance represents, in a formal manner, different relationships between entities (data items), activities (data transitions) and agents (which cause the transition). In other words, it can be understood as a formal set of documents which help in understanding the data existence and its flow to trace its integrity (quality). \cite{b1}
%
%Information flow tracking is a security mechanism designed to monitor how sensitive information spreads in a system. In data provenance context, information flow tracking is used to track the data and is be further used to put limits on the dissemination of a piece of sensitive data once it's out of its container of information. This allows high-level policies such as " my banking information will not be sent outside my system " or " my banking information will not be mixed with my wife's banking information" to be enforced easily.  \cite{b2}. Explicit information flow is defined as the copy, usually partial, of the content of one container of information to another. \cite{b3}
%
%Data provenance with a completeness property ensures that all the information flows of the data within the system are being recorded. Pasquier et. al \cite{b6} in his work on Run-time Analysis of Whole-system provenance ensures the completeness and accuracy of the provenance capture mechanism. He further makes use of Georget et al's \cite{b3} formalism to show that all the information flows between the kernel objects are properly recorded. However, this does not prove that all the security-related flows are also being captured by the system. This also does not tell us if the completeness of the information-flow property suffices to find intrusions.
%
%
%\begin{figure}
%	\centering
%	\includegraphics[width=0.7\linewidth]{Architecture-diagram}
%	\caption[]{High-level Architectural view. A structured way to formalize whole-system data provenance is to formalize if all the information flows are being captured. In order to make sure that all information flows are being captured we need to make use of LSM to insert hooks at every point in the kernel where a user-level system call is about to result in access to an important internal kernel object.}
%	\label{fig:architecture-diagram}
%\end{figure}
%
%
%
%
%
%
%
%Hence, as a part of this work we try to answer the question about complete and security-related information flows of the system using a whole-system provenance capture called Camflow.\cite{b7}
%
%The first part of the challenge includes answering the question if the completeness formalism is enough to detect all security related flows. Xueyuan et al. use a whole-system provenance capture tool for fault-detection \cite{b8}. Pasquier et. al. \cite{b6} shows that it's possible to detect intrusions during run-time of the system using CamFlow. However, is it enough to detect all sorts of data leaks and/or intrusions is one question we try to answer as a part of this work? This analysis has been performed for Linux Kernel 4.3. We try to see if the formalism still holds for the updated kernel version (5.0.x).
%
%The second contribution of our work is if indeed Camflow \cite{b7}  is able to capture all security-related flows, do they all show up as an anomaly in the provenance-graph. The provenance-graph is constructed from the data captured from all the security-related flows in the system. 
%
%
%\begin{figure}
%	\centering
%	\includegraphics[width=0.7\linewidth]{DAG}
%	\caption{A simple provenance DAG: a process P sends packet Pkt(a) to process Q using the sockets S and T.}
%	\label{fig:dag}
%\end{figure}
%
%\section{Background}
%We conduct an analysis to see if the current version of the information flow patch applied to CamFlow is capable enough to catch all security violations. In order to do so, we make use of three open-source tools/libraries/patches CamFlow, GraphChi and complete flow capture formalism proposed by Georget et al \cite{b3}. 
%
%\subsection{CamFlow} 
%
%As we have discussed before data provenance records the chronology of ownership, change, and movement of an object or a resource. There are many provenance capture systems built before CamFlow such as PASS \cite{b10}, Hi-Fi\cite{b11} and Linux Provenance Module (LPM) \cite{b12}. However, none of the above mentioned Operating systems provenance capture systems utilize LSM except for it's predecessor LPM. LSM is a framework that allows the kernel to support a variety of computer security models while avoiding favoritism towards any of the security implementations. 
%
%We chose CamFlow for testing our hypothesis because it adopts the LSM architecture and support for NetFilters which makes it a maintainable practical whole-system provenance implementation.CamFlow maintains an information-flow patch based on Georget et al's \cite{b3}formalism which provides completeness guarantees. The fact that Cam-flow provides completeness guarantees, its a strong choice for researchers to use it as a fault-detection tool \cite{b8}. Pasquier et al. use CamFlow to perform a run-time analysis of whole-system provenance to find intrusions \cite{b6}. However, since the completeness guarantees are specific to kernel version 4.3, we want to provide a formalism which shows that it holds for the updated kernel versions.  
%
%In order to perform such an analysis,  Pasquier et al. have maintained a systematic build script for CamFlow Linux provenance.  This script generates kernel patch for CamFlow Linux provenance capture. It consists of two parts which are running the automated Travis script and  CircleCI script.  The Travis script 
%
%builds the kernel and eventually builds the kernel patch. The CircleCl script performs kernel analysis.
%
%\subsection{GraphChi}
%
%The representation of the provenance data is in the form of a directed acyclic graph (DAG). Every node in the DAG represents an entity, an activity or an agent. Each directed edge represents an interaction between each node. Fig.2 represents a simple example. In our context, entities are kernel objects, activities are tasks, and agents are users and groups. Fig 2 represents packet being sent from process P to Q from socket S to T. 
%
%We use GraphChi, a vertex-centric graph processing model to generate program models and to generate anomalies. The purpose of utilizing GraphChi is to answer that in case of security violations in the system, are the anomalies reflected in the graphs. This is a tool which we use for ensuring if all anomalies are reflected on the provenance graph after the capture. 
%
%
%\subsection{Formalism of complete mediation}
%
%Georget et al. \cite{b3} proposed a formalism to verify the property for complete capture of the information flows called \textit{Complete Mediation}. The authors use a compiler-assisted and reproducible static analysis on Linux kernel to verify that the LSM hooks are correctly placed with respect to operations generating information flows. This ensures that LSM-based information flow monitors can properly track all information flows.
%
%
%
%
%\section{Evaluation}
%
%The evaluation of the system is performed in two parts. The first part is where we focus on the completeness and accuracy of the current CamFlow patch with respect to the new kernel version. We also try to find if all the security-related flows are being captured by the LSM-interface. The second half of our evaluation focuses on the anomaly detection aspect. We want to prove ( or disprove ) if all security related flows are being captured by the LSM-interface, then the anomaly must show up in the provenance graph.
%
%\subsection{LSM interface and security-related flows}
%Before analysing if all security-related flows are being captured by the interface, we first ensure the sompletness and accuracy of the interface
%
%
%
%
%
%\subsubsection{Completeness}
%We want to ensure that all flows of information between kernel objects are properly recorded. The LSM framework \cite{b14} was originally implemented to support Mandatory Access Control (MAC) schemes but not information flow tracking. Georget et al \cite{b3} through static analysis of the kernel code base demonstrated that LSM framework is applicable to information flow tracking. By adding a small number of hooks it is possible to properly intercept all information flows between kernel objects.
%
%In order to run the analysis we use the scripts written by Pasquier et al \cite{b14} which generates the coverage, hooks, relations, stats and vertices for the system. We can see that some of the hooks such as the ones listed in Table 1 are ignored. The automated CircleCl script performs kernel source code analysis and generates the above mentioned attributes in the docs folder. 
%
%
%
%
%From the analysis we observe that the coverage is not complete. Hence, the attacker can still find ways to attack the system. The coverage for different system calls is different. For eg. we observe that for \_\_x64\_sys\_open system call the coverage is 8 out of 12. This means that out of the 12 hooks which were originally called by this system call, only 8 hooks were implemented. We observe similar coverage for different system calls. 
%
%\subsubsection{Security-related flows}
%Based on the coverage we observe a couple of security-related flows which will not be captured by the current provenance capturing mechanism.  If the attack is strictly about extracting information from within an application memory, the effect is not observed. 
%
%\subsection{Anomaly detection}
%Now that we are aware that some attacks are caught by the capture mechanism and there are some attacks which are not. We try to observe if all the attacks whose flow is being recorded by the capture mechanism if it is reflected on the graph. 
%
%From our evaluation, we understand that if the flow is being recorded it does get reflected on the graph. This is observed from the coverage analysis which we obtained from running the call graph scripts. However, currently, there is no formalism to prove so. So an important and very relevant contribution to this work would be to provide some form of formalism to show that this happens. Hence, my goal is to show some form of formalism before the final submission.  
%\section{Conclusion}
%
%After analyzing the results we observe a couple of things about CamFlow architecture, the formalism proposed by Georget et al. and finally about LSM. 
%
%The reason we performed this analysis was to observe the changes which will be required in the CamFlow code-base to make sure it adheres to the latest versions of the kernel. The other reason we carried out this analysis was how feasible is it to maintain CamFlow with respect to the latest kernel releases. The ultimate goal is to have provenance integrated into the mainline Linux kernel. Thus if we can keep up with the versions, it shows that CamFlow is a fully self-contained Linux kernel module. 
%
%Another observation from the results we can notice a clear completeness-security gap. This tells us that there is a lot more work to be done in order to build intrusion detection systems using provenance capture mechanisms. Provenance capture mechanisms can capture only a subset of the possible attacks.
%
%\section* {Discussion Topics}
%The first interesting discussion point is understanding the completeness-security gap if any. Currently, the DAGs are generated with respect to the provenance data. However, this is not taking the timing aspect into consideration. Provenance capture mechanisms currently are data related. The way we can capture anomalies is by learning the data pattern from previous immutable data. However, if the attacker brings in the timing aspect, keeping the data intact can that cause damage? It depends on the application if data delays can cause harm. 
%
%Another interesting point to discuss would be the completeness of information flows means that it guarantees that all flows are being captured. So you need to place hooks at the relevant places which ensures that all information flow paths are being recorded. However, again this does not cover the event aspect of the system. What if instead of modifying the data with alone or modifying the data with respect to time, it modifies the event in a smart way such that the correlations created by the DAG cannot spot it.  Instead of sending the data from process P to process Q, it sends it to process R. 
%\section*{Acknowledgment}
%
%This work was done as a part of our term project for CPSC 508. CPSC 508 is a graduate level operating systems course taken by Dr. Margo Seltzer.

\begin{thebibliography}{00}
\bibitem{b1} Lucian Carata, Sherif Akoush, Nikilesh Balakrishnan, Thomas Bytheway,
Ripduman Sohan, Margo Seltzer, Andy Hopper, an ``A Primer on Provenance,'' acmqueue, 2014.
\bibitem{b2} Daniel Crawl and Ilkay Altintas , A Provenance-Based Fault Tolerance Mechanism for
Scientific Workflows.

\bibitem{b3} Laurent Georget, Mathieu Jaume, Guillaume Piolle, ``Verifying the reliability of operating system-level information flow control systems in linux,'' Proceedings of the 5th International FME Workshop on Formal Methods in Software Engineering, FormaliSE ’17, pages 10–16, Piscataway, NJ, USA, 2017. IEEE Press.

\bibitem{b4}GERWIN KLEIN, ``Operating system verification—An overview,"Sadhan ¯ a¯ Vol. 34, Part 1, February 2009, pp. 27–69. © Printed in India

\bibitem{b5} Jonathan Pincus and Brandon Baker , ``Mitigations for Low-Level Coding Vulnerabilities:
Incomparability and Limitations ,'' 2004.


\bibitem{b6}Thomas Pasquier, Xueyuan Han, Thomas Moyer, Adam Bates, Olivier Hermant, David Eyers, Jean Bacon, Margo Seltzer, ``Runtime Analysis of Whole-System Provenance
,''16 pages, 12 figures, 25th ACM Conference on Computer and Communications Security 2018.



\bibitem{b7} Thomas Pasquier, Xueyuan Han, Mark Goldstein, Thomas Moyer, David Eyers, Margo Seltzer, Jean Bacon, Practical Whole-System Provenance Capture
, SoCC '17 Proceedings of the 2017 Symposium on Cloud Computing.


\bibitem{b8} Xueyuan Han, Thomas Pasquier, Tanvi Ranjan, Mark Goldstein, and Margo Seltzer, Harvard University

 , ``FRAPpuccino: Fault-detection through Runtime Analysis of Provenance ,'' HotCloud'17.



\bibitem{b9} PASQUIER, T. F.-M., SINGH, J., BACON, J., AND EYERS, D. , ``Information flow audit for paas clouds.
Cloud Engineering
(IC2E), 2016 IEEE International Conference on (2016), IEEE,
pp. 42–51.


\bibitem{b10} MUNISWAMY-REDDY, K.-K., HOLLAND, D. A., BRAUN, U.,
AND SELTZER, M. I. , `` Provenance-aware storage systems. ,'' In
USENIX Annual Technical Conference, General Track (2006),
pp. 43–56..


\bibitem{b11} POHLY, D. J., MCLAUGHLIN, S., MCDANIEL, P., AND BUTLER, K , ``Hi-fi: collecting high-fidelity whole-system provenance ,'' In Proceedings of the 28th Annual Computer Security Applications Conference (2012), ACM, pp. 259–268.


\bibitem{b12} Adam Bates, Dave (Jing) Tian, and Kevin R.B. Butler , ``Trustworthy Whole-System Provenance
for the Linux Kernel.
24th USENIX Security Symposium, 2015.








\bibitem{b13} PASQUIER, T. , ``Camflow information flow patch.
In https://github. com/CamFlow/information- flow- patch.
.


\bibitem{b14} 	Chris Wright,	
Crispin Cowan,	
Stephen Smalley,	
James Morris,	
Greg Kroah-Hartman.	
, ``Linux Security Modules: General Security Support for the Linux Kernel.
Proceeding
Proceedings of the 11th USENIX Security Symposium
Pages 17-31 

August 05 - 09, 2002 .


\bibitem{b15} PASQUIER, T. , ``CamFlow development.
In https://github.com/CamFlow/camflow-dev.

\bibitem{b16} INRIA , ``The Kayrebt Toolset
In http://kayrebt.gforge.inria.fr/


17 Peter Buneman, Sanjeev Khanna, and Tan Wang-Chiew. 2001. Why and where:
A characterization of data provenance. In International Conference on Database
Theory. Springer, 316–330.


18 Allison Woodruff and Michael Stonebraker. 1997. Supporting fine-grained data
lineage in a database visualization environment. In International Conference on
Data Engineering. IEEE, 91–102.

19 Khalid Belhajjame, Reza B’Far, James Cheney, Sam Coppens, Stephen Cresswell,
Yolanda Gil, Paul Groth, Graham Klyne, Timothy Lebo, Jim McCusker, Simon
Miles, James Myers, Satya Sahoo, Luc Moreau, and Paolo et al. Missier. 2013.
Prov-DM: The PROV Data Model. Technical Report. World Wide Web Consortium
(W3C). https://www.w3.org/TR/prov-dm/

20  Rastislav Bodík, Rajiv Gupta, and Mary Lou Soa. 1997. Refining Data Flow
Information Using Infeasible Paths. SIGSOFT Software Engineering Notes 22, 6
(Nov. 1997).

21 Richard Matthew Stallman and the GCC developer community. 2013. Using
the GNU Compiler Collection (GCC). Technical Report. https://gcc.gnu.org/
onlinedocs/gcc-4.8.4/gcc/

22 author="Xiao, Jidong
and Huang, Hai
and Wang, Haining",
editor="Thuraisingham, Bhavani
and Wang, XiaoFeng
and Yegneswaran, Vinod",
title="Kernel Data Attack Is a Realistic Security Threat",
booktitle="Security and Privacy in Communication Networks",
year="2015",
publisher="Springer International Publishing"/

23 R. M. Stallman and the GCC developer community, “Using the
GNU Compiler Collection (GCC),” Tech. Rep., 2013. [Online].
https://gcc.gnu.org/onlinedocs/gcc-4.8.4/gcc/

24 D. Beyer, T. A. Henzinger, R. Jhala, and R. Majumdar, “The software
model checker Blast,” International Journal on Software Tools for
Technology Transfer, vol. 9, no. 5, 2007.

25 WireX Communications. Linux Security Module. http://
lsm.immunix.org/, April 2001.

26 Stephen Smalley, Timothy Fraser, and Chris Vance. Linux Security Modules: General Security Hooks for Linux. http:
//lsm.immunix.org/, September 2001.

27 Neil Brown. Sparse : a look under the hood. Linux Weekly News. 8 juin 2016.
url : https://lwn.net/Articles/689907/ (visité le 01/02/2017) (cf. p. 29).

28 Yoann Padioleau, Julia L. Lawall, René Rydhof Hansen et Gilles Muller.
« Documenting and Automating Collateral Evolutions in Linux Device Dri-
vers ». In : European Conference on Computer Systems (EuroSys 2008). Glas-
gow, Scotland : ACM, avr. 2008, p. 247–260 (cf. p. 29).

29 Junfeng Yang, Ted Kremenek, Yichen Xie et Dawson Engler. « MECA : an
extensible, expressive system and language for statically checking security
properties ». In : ACM conference on Computer and Communications Security
(CCS 2003). Washington D.C., USA : ACM, oct. 2003, p. 321–334 (cf. p. 29).

30Jeffrey Scott Foster. « Type qualifiers : lightweight specifications to improve
software quality ». Thèse de doct. Berkeley, CA, USA : University of California
at Berkeley, 2002 (cf. p. 29).

31 http://kayrebt.gforge.inria.fr/

32 Christophe Hauser. « Détection d’intrusion dans les systèmes distribués par
propagation de teinte au niveau noyau ». Doctoral thesis. Rennes, France :
University of Rennes 1, juin 2013

33 ] K. Muniswamy-Reddy, D. A. Holland, U. Braun, and M. Seltzer.
Provenance-Aware Storage Systems. In Proceedings of the 2006
USENIX Annual Technical Conference, 2006.

34 J. Seibert, G. Baah, J. Diewald, and R. Cunningham. Using
Provenance To Expedite MAC Policies (UPTEMPO) (Previously
Known as IPDAM). Technical Report USTC-PM-015, MIT Lincoln Laboratory, October 2014.

35 A. Bates, K. Butler, A. Haeberlen, M. Sherr, and W. Zhou. Let
SDN Be Your Eyes: Secure Forensics in Data Center Networks.
In NDSS Workshop on Security of Emerging Network Technologies, SENT, Feb. 2014.

36 A. Gehani, B. Baig, S. Mahmood, D. Tariq, and F. Zaffar. Finegrained Tracking of Grid Infections. In Proceedings of the
11th IEEE/ACM International Conference on Grid Computing,
GRID’10, Oct 2010.

37 D. Tariq, B. Baig, A. Gehani, S. Mahmood, R. Tahir, A. Aqil, and
F. Zaffar. Identifying the Provenance of Correlated Anomalies. In
Proceedings of the 2011 ACM Symposium on Applied Computing,
SAC ’11, Mar. 2011.

38 D. Nguyen, J. Park, and R. Sandhu. Dependency Path Patterns
As the Foundation of Access Control in Provenance-aware Systems. In Proceedings of the 4th USENIX Conference on Theory
and Practice of Provenance, TaPP’12, pages 4–4, Berkeley, CA,
USA, 2012. USENIX Association.

39 ] Q. Ni, S. Xu, E. Bertino, R. Sandhu, and W. Han. An Access
Control Language for a General Provenance Model. In Secure
Data Management, Aug. 2009

40 J. Park, D. Nguyen, and R. Sandhu. A Provenance-Based Access
Control Model. In Proceedings of the 10th Annual International
Conference on Privacy, Security and Trust (PST), pages 137–144,
2012.

41. Nickolai Zeldovich, Silas Boyd-Wickizer, Eddie Kohler, and David Mazières.
2006. Making information flow explicit in HiStar. In Symposium on Operating
Systems Design and Implementation (OSDI’06). USENIX Association, 263–278.

42 Maxwell Krohn, Alexander Yip, Micah Brodsky, Natan Cliffer, M Frans Kaashoek,
Eddie Kohler, and Robert Morris. 2007. Information flow control for standard OS
abstractions. In ACM SIGOPS Operating Systems Review, Vol. 41. ACM, 321–334

43 Adwait Nadkarni, Benjamin Andow, William Enck, and Somesh Jha. 2016. Practical DIFC enforcement on Android. In USENIX Security Symposium. 1119–1136

44 Stephen Smalley, Chris Vance, and Wayne Salamon. 2001. Implementing SELinux
as a Linux security module. NAI Labs Report 1, 43 (2001), 139.
\end{thebibliography}




\end{document}
